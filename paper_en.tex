\documentclass{article}
\usepackage[utf8]{inputenc}
\usepackage[english]{babel}
\usepackage[legalpaper, portrait, margin=1in]{geometry}
\usepackage{setspace}
\onehalfspacing
\usepackage{parskip}
\setlength{\parindent}{0cm}
\usepackage{enumitem}
\setlist{nosep}
\usepackage{hyperref}
\usepackage{graphicx}

\hypersetup{
  	colorlinks,
	citecolor=black,
	filecolor=black,
	linkcolor=black,
	urlcolor=black
}

\title{%
  Olympus Protocol \\
  \large A scalable peer-to-peer electronic cash system}
\author{
  Berrueta, Enrique\\
  \texttt{eabz@polispay.org}
  \and
  Bustos, Ricardo\\
  \texttt{eros@polispay.org}
  \and
  Meyer, Julian\\
  \texttt{julianmeyer2000@gmail.com}
}
\date{June 2020}

\begin{document}

	\maketitle
	\begin{abstract}
		The cryptocurrency sector is continuously evolving, new consensus mechanisms, algorithms, and cryptographic proofs have appeared in the blockchain development sector. Olympus introduces a unique consensus system that focuses on making the network scalable, secure, and user-friendly without losing the usability of a decentralized system. The protocol introduces a new organized consensus mechanism based on Ethereum CASPER, a modularized transaction system based on dynamic payloads using bilinear pairing systems for cryptographic proofs.
	\end{abstract}
	
	\newpage
	
    \tableofcontents		
	
	\newpage
	
	\section{Introduction}
	\section{Background}
		\subsection{Blockchains}
		\subsection{Byzantine Fault}
		\subsection{Double Spending}
		\subsection{Replicated State Machines}
		\subsection{Consensus}
		\subsection{Ethereum CASPER}
		
	\section{Bilinear pairing cryptography}
	    \subsection{Boneh–Lynn–Shacham signature scheme}	
	    \subsection{Signature aggregation}	
	    
	\section{Consensus}
	    \subsection{CASPER Fundamentals}	
	    \subsection{Validators}		    
	    \subsection{Deposits}	
	    \subsection{Exits}		    
	    \subsection{Slashing}	
	    
	\section{Supply}
	    \subsection{Emission}	
	    \subsection{Incentivation}		    
	    
	\section{DAO}
	    \subsection{Funcional DAO for Open Source Projects}	
	    \subsection{DAO Keys}		    
	    \subsection{DAO Signatures and Vote mechanism}		    
	    \subsection{Community participation}		    

	
\end{document}