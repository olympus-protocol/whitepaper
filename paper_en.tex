\documentclass{article}
\usepackage[utf8]{inputenc}
\usepackage[english]{babel}
\usepackage[letterpaper, total={6in, 8in}]{geometry}
\usepackage{setspace}
\onehalfspacing
\usepackage{parskip}
\usepackage{csquotes}
\usepackage{biblatex}
\addbibresource{references.bib}
\usepackage{hyperref}
\hypersetup{
  	colorlinks,
	citecolor=black,
	filecolor=black,
	linkcolor=black,
	urlcolor=black
}
\setlength{\parindent}{0cm}
\title{%
  Olympus Protocol \\
  \A scalable peer-to-peer electronic cash system}
\author{
  Berrueta, Enrique\\
  \texttt{eabz@polispay.org}
  \and
  Bustos, Ricardo\\
  \texttt{eros@polispay.org}
  \and
  Meyer, Julian\\
  \texttt{julianmeyer2000@gmail.com}
}
\date{June 2020}

\begin{document}

	\maketitle
	\begin{abstract}
		The cryptocurrency sector is continuously evolving, new consensus mechanisms, algorithms, and cryptographic proofs have appeared in the blockchain development sector. Olympus introduces a unique consensus system that focuses on making the network scalable, secure, and user-friendly without losing the usability of a decentralized system. The protocol introduces a new organized consensus mechanism based on Ethereum CASPER, a modularized transaction system based on dynamic payloads using bilinear pairing systems for cryptographic proofs.
	\end{abstract}
	
	\newpage
	\tableofcontents
	\newpage
	
	\newpage
	
	\printbibliography
	
\end{document}